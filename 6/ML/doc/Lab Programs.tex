\documentclass{article}

% package
\usepackage[a3paper,margin=.5in]{geometry}
\usepackage{enumitem}
\usepackage{amsmath,amssymb}
\usepackage{pgffor}
\usepackage{xcolor}
\usepackage{graphicx}
\usepackage{minted}
\usepackage{hyperref}

% detailes
\author{koku17}
\title{Machine Learning Lab}

% presets
\hypersetup{
	hidelinks
}

\setminted{
	breaklines,
	tabsize=4,
	breakanywhere=true
}

% ricing
\def \darkmode{1}

\if\darkmode1
    \pagecolor{black}
    \color{white}
    \usemintedstyle{one-dark}
\else
    \usemintedstyle{perldoc}
\fi

% macros
\def \question#1{\item #1 \addcontentsline{toc}{section}{Lab \arabic{enumi}}}
\def \answer{\item [$\rightarrow$]}

\begin{document}
	\pagenumbering{gobble} \maketitle \newpage
	\pagenumbering{roman} \pdfbookmark[1]{Contents}{} \tableofcontents \newpage
	\pagenumbering{arabic}

	\begin{enumerate}[label=\arabic*. ]
		\question{Develop a program to}
			\begin{itemize}
				\item Load a dataset and select one numerical column.
				\item Compute mean, median, mode, standard deviation, variance, and range for a given numerical column in a dataset.
				\item Generate a histogram and boxplot to understand the distribution of the data.
				\item Identify any outliers in the data using IQR.
				\item Select a categorical variable from a dataset.
				\item Compute the frequency of each category and display it as a bar chart or pie chart.
			\end{itemize}
		\answer \inputminted{python}{../Lab1.py} \newpage

		\question{Develop a program to}
			\begin{itemize}
				\item Load a dataset with at least two numerical columns (e.g., Iris, Titanic).
				\item Plot a scatter plot of two variables and calculate their Pearson correlation coefficient.
				\item Write a program to compute the covariance and correlation matrix for a dataset.
				\item Visualize the correlation matrix using a heatmap to know which variables have strong positive / negative correlations.
			\end{itemize}
		\answer \inputminted{python}{../Lab2.py} \newpage

		\question{Develop a program to implement Principal Component Analysis (PCA) for reducing the dimensionality of the Iris dataset from 4 features to 2.}
		\answer \inputminted{python}{../Lab3.py} \newpage

		\question{Develop a program to}
			\begin{itemize}
				\item Load the Iris dataset.
				\item Implement the $k$-Nearest Neighbors ($k$-NN) algorithm for classifying flowers based on their features.
				\item Split the dataset into training and testing sets and evaluate the model using metrics like accuracy and F1-score.
				\item Test it for different values of $k$ (e.g., $k=1,3,5$) and evaluate the accuracy.
				\item Extend the $k$-NN algorithm to assign weights based on the distance of neighbors (e.g., $\text{weight}=\frac{1}{d^2}$).
				\item Compare the performance of weighted $k$-NN and regular $k$-NN on a synthetic or real-world dataset.
			\end{itemize}
		\answer \inputminted{python}{../Lab4.py} \newpage

		\stepcounter{enumi}
		\question{
			Implement the non-parametric Locally Weighted Regression algorithm in order to fit data points. \\
			Select appropriate data set for your experiment and draw graphs.
		}
		\answer \inputminted{python}{../Lab6.py} \newpage

		\question{Develop a program to}
			\begin{itemize}
				\item Demonstrate the working of Linear Regression and Polynomial Regression.
				\item Use Boston Housing Dataset for Linear Regression and Auto MPG Dataset (for vehicle fuel efficiency prediction) for Polynomial Regression.
			\end{itemize}
		\answer \inputminted{python}{../Lab7.py} \newpage

		\question{Develop a program to}
			\begin{itemize}
				\item Load the Titanic dataset.
				\item Split the data into training and test sets.
				\item Train a decision tree classifier.
				\item Visualize the tree structure.
				\item Evaluate accuracy, precision, recall, and F1-score.
			\end{itemize}
		\answer \inputminted{python}{../Lab8.py} \newpage

		\question{
			Develop a program to implement the Naive Bayesian classifier considering Iris dataset for training. \\
			Compute the accuracy of the classifier, considering the test data.
		}
		\answer \inputminted{python}{../Lab9.py}
	\end{enumerate}
\end{document}
