\begin{enumerate}[label=\arabic*. ]
	\item Explain what is an algorithm in computing ?
	\answer
		\begin{itemize}[label=$\ast$]
			\item An algorithm is a well-defined computational procedure that takes some value as
				input and generates some value as output
			\item In simple words, it's a sequence of computational steps that converts input into
				the output
		\end{itemize}

	\item Explain what is time complexity of Algorithm ?
	\answer
		\begin{itemize}[label=$\ast$]
			\item Time complexity of an algorithm indicates the total time needed by the program to
				run to completion
			\item It is usually expressed by using the big O notation
		\end{itemize}

	\item Explain the greedy method
	\answer
		\begin{itemize}[label=$\ast$]
			\item Greedy method is the most important design technique, which makes a choice that
				looks best at that moment
			\item A given 'n' inputs are required us to obtain a subset that satisfies some
				constraints that is the feasible solution
			\item A greedy method suggests that one can device an algorithm that works in stages
				considering one input at a time
		\end{itemize}
	
	\item Specify the algorithms used for constructing Minimum cost spanning tree
		\begin{itemize}[label=$\ast$]
			\item Prim's Algorithm
			\item Kruskal's Algorithm
		\end{itemize}
	
	\item State Kruskal Algorithm
	\answer The algorithm looks at a MST for a weighted connected graph as an acyclic subgraph with $|v|-1$
		edges for which the sum of edge weights is the smallest
	
	\item Difference between Prim's and Kruskal's Algorithm
	\answer ~ \vspace{-2em}
		\begin{center}
			\begin{tabular}{|p{.4\columnwidth}|p{.4\columnwidth}|} \hline
				\multicolumn{1}{|c|}{Prim's Algorithm} &
				\multicolumn{1}{|c|}{Kruskal's Algorithm} \\ \hline
				The tree that we are making or growing always remains connected &
				The tree that we are making or growing usually remains disconnected \\ \hline
				Prim's Algorithm grows a solution from a solution from a random vertex
				by adding the next cheapest vertex to the existing tree &
				Kruskal's Algorithm grows a solution from the cheapest edge to the next cheapest
				edge to the existing tree/forest \\ \hline
				Prim's algorithm is faster for dense graphs &
				Kruskai's algorithm is faster for spars graphs \\ \hline
			\end{tabular}
		\end{center}

	\item How efficient is the Kruskal algorithm ?
	\answer
		\begin{itemize}[label=$\ast$]
			\item Kruskal performs better in typical situations (sparse graphs) because it uses
				simpler data structures and its time complexity is
				$\text{O}(\text{E}\log\text{V})$
			\item Where E is the number of edges \\ V is number of vertices
		\end{itemize}
	
	\item What is a minimum spanning tree ?
	\answer A minimum spanning tree is a subset of the edges of a connected, undirected graph that connects
		all the vertices together with the minimum possible total weight
	
	\item Can you explain what a graph is in the context of computer science and informatics ?
	\answer
		\begin{itemize}[label=$\ast$]
			\item A graph is a collection of nodes, also called vertices, and the edges that connect
				them
			\item Graphs can be used to model many different types of relationships, including those
				found in social networks, transportation systems, and computer networks
		\end{itemize}
	
	\item What is the difference between weighted and unweighted graphs ?
	\answer
		\begin{itemize}[label=$\ast$]
			\item A weighted graph is a graph in which each edge has a weight or cost associated
				with it
			\item An unweighted graph is a graph in which the edges do not have weights associated
				with them
		\end{itemize}
	
	\item When would you use Prim's algorithm over Kruskal's algorithm ?
	\answer
		\begin{itemize}[label=$\ast$]
			\item Prim's algorithm is used when you want to find the minimum spanning tree of a
				graph that is not necessarily connected
			\item Kruskal's algorithm is used when you want to find the minimum spanning tree of a
				graph that is already connected
		\end{itemize}
	
	\item What are the disadvantages of using Kruskal's algorithm on large graphs with many edges ?
	\answer
		\begin{itemize}[label=$\ast$]
			\item Kruskal's algorithm can be slow on large graphs with many edges because it has to
				consider all of the edges in the graph in order to find the minimum spanning
				tree
			\item This can be time-consuming, especially if the graph is very large
			\item Additionally, Kruskal's algorithm can be difficult to implement on some types of
				graphs
		\end{itemize}
	
	\item Can you give me examples of real-world applications where minimum spanning trees can be used ?
	\answer
		\begin{itemize}[label=$\ast$]

			\item One example of where minimum spanning trees can be used is in network design
			\item If you are trying to connect a set of computers or other devices together with the
				minimum amount of cable or other wiring, then you can use a minimum spanning
				tree algorithm to find the most efficient way to do so
			\item Another example is in cluster analysis, where you might use a minimum spanning
				tree to group together similar items in a dataset
		\end{itemize}
\end{enumerate}
